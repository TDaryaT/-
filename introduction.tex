\section*{Введение}

\addcontentsline{toc}{section}{Введение}
Слежение за объектами -- одна из важных и интересных задач, применяемая во многих сферах жизни общества: в навигации, видео-наблюдении, анализе движения объекта, компьютерного зрения, аналитике спортивных видео материалов.  

В данный момент существует множество моделей слежения за объектами (их также называют трекерами), но не смотря на это, проблема точности, стабильности и быстродействия отслеживания все еще существует. Часто, возникают проблемы, связанные с отслеживанием в режиме реального времени, окклюзиями (затруднениями в разборе изображения), затемнением изображения, резкими движениями, изменениями освещения, уходом объекта из кадра, загромождением фона.

Многие, уже существующие методы решения, умеют неплохо справляться с возникающими проблемами, однако, они пользуются чрезмерным усложнением модели и требуют больших вычислений, из-за чего реализация в режиме реального времени становится затруднительной. Также, многие методы требуют больших затрат памяти для своей работы, в частности большее место для работы сети необходимо выделить для матрицы перехода или, как ее еще называют, матрицы весовых коэффициентов. Данные сведения подтверждают актуальность выбранной темы. 

Использование квантования весов в нейронной сети для изменения ее весовых коэффициентов  является одним из многочисленных решений проблемы, связанной с затратами памяти программной реализации. В свою очередь, данное внедрение делает актуальной задачу исследования влияния числа уровней квантования весов. 

Целью данной работы является исследование, разработка и модификация существующего алгоритма слежения за движущимся объектом на наборе изображений в режиме реального времени.

Задачи, которые были поставлены в данной работе:

\begin{itemize}[leftmargin=0em, itemindent=2.5 em,itemsep=1.5 pt,parsep=1.5 pt]
        \item[--] исследование существующих алгоритмов слежения за объектами; 
        \item[--] анализ существующего алгоритма сети глубокого доверия для слежения за конкретным объектом;
        \item[--] использование задачи квантования весовых коэффициентов на выбранной сети глубокого обучения;
        \item[--] реализация алгоритма квантования весов сети, которая является, в свою очередь, модификацией данной сети;
        \item[--] изучение влияния числа уровней квантования весовых коэффициентов;
        \item[--] изучение влияния квантования весов на алгоритм слежения за объектом.
\end{itemize}
