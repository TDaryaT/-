\section*{Реферат}

\pagestyle{empty}
Название работы: Слежение за объектом сетями глубокого доверия.

Работа содержит 35 страниц, 10 иллюстраций, 3 таблицы, 10 источников, 5 приложений.

Ключевые слова: слежение за объектом, квантование весов, сеть глубокого доверия, число уровней квантования, нейрокомпьютерная сеть.

Объектом исследования является сеть глубокого доверия для слежения за объектом. 

Цель работы: исследование, разработка и модификация существующего алгоритма слежения за движущимся объектом на наборе изображений в режиме реального времени.

В ходе работы используется MATLAB версии R2019b9.7, исследуется выбранная сеть глубокого доверия DLT, применяется алгоритм квантования весовых коэффициентов матрицы и приводится аналитика влияния алгоритма на слежение за объектом. 

В результате получено число уровней квантования $L=128, 256$ необходимое для алгоритма квантования и приемлемого слежения. Также, зафиксировано уменьшение памяти в 10 раз для матрицы весовых коэффициентов.

Областью применения задачи слежения за объектом является навигация, видео-наблюдение, анализ движения и компьютерное зрение.
