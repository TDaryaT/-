\section*{Заключение}
\addcontentsline{toc}{section}{Заключение}

В данной работе был описан эффективный алгоритм распознавания траектории движения объекта на видео-последовательности. С его помощью можно осуществлять отслеживание объектов эффективнее, чем, например, с помощью органов зрения. Помимо этого получилось улучшить существующий алгоритм слежения. 

В ходе работы, проделано следующее:

\begin{itemize}[leftmargin=0em, itemindent=2.5 em,itemsep=1.5 pt,parsep=1.5 pt]

    \item[--] рассмотрены аналоги, решающие задачу отслеживания объекта на наборе изображений;
    \item[--] исследован эффективный алгоритм DLT для работы с отслеживанием траектории движения на видеоматериале;
    \item[--] использована задача квантования весов для улучшения работы алгоритма;
    \item[--] приведена реализация алгоритма квантования весов для конкретной нейронной сети;
    \item[--] зафиксировано уменьшение занимаемой памяти после выполнения алгоритма квантования весовых коэффициентов;
    \item[--] рассмотрено влияние квантования весов на временную составляющую алгоритма;
    \item[--] рассмотрено влияние квантования весов на смещение центра отслеживаемого объекта;
    \item[--] проведена сравнительная аналитика уровней квантования весовых коэффициентов;
    \item[--] были получены приемлемые числа уровней квантования.
\end{itemize}

Таким образом, были решены все поставленные выше задачи и достигнута цель данной работы. В свою очередь, данный проект может иметь дальнейшее перспективы развития. 

В работе был рассмотрен показатель улучшения памяти, но в данной области остаются и другие проблемы связанные с точным обнаружением отслеживаемого объекта. Замечено, что особенно на сложных последовательностях изображений чаще всего происходит затруднение, такое как попадание в тень объекта или выход из кадра. Модификация основного алгоритма решающая эту проблематику позволит значительно улучшить данный алгоритм.

Также, данная программа тестировалась на достаточно устаревшем устройстве, в силу чего могли быть достигнуты более хорошие результаты, особенно временные. 

В настоящее время, реализация алгоритмов нейронных сетей зачастую происходит на языке программирования Python, в котором реализованы многие оптимизированные пакеты функций, благодаря которым можно получить выигрыш по тем или иным показателям; таким образом возможно осуществить оптимизацию существующего кода программы. 